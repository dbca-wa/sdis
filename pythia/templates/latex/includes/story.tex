
\chapterimage{{ STATIC_ROOT }}/img/80mile.jpg}
\chapter*{Students name and track Pilbara turtles}
\addcontentsline{toc}{chapter}{Students name and track Pilbara turtles}
\label{chap:story}
\chaptermark{}


LeBron and Sandy Shellington are just two of the creative names school students
from Karratha and Ashburton have given a group of nesting marine turtles. The
students were invited to name and track the female turtles as part of a Parks
and Wildlife monitoring program.

\begin{wrapfigure}{r}{0.4\textwidth}%
\includegraphics[width=\linewidth]%
    {{ STATIC_ROOT }}/img/storytime.jpg}
\caption*{Installing a satellite TV on a flatback turtle.}
\end{wrapfigure}

The joint program, between Regional and Fire Management Services and Science and
Conservation Division's Marine Science Program, aims to gain a better
understanding of marine turtles off the North West Shelf and greater Western
Australian waters.

As part of the project, Parks and Wildlife staff and volunteers visited
Montebello Islands Marine Park, where they fitted satellite transmitters to 12
flatback turtles, five green turtles, five hawksbill turtles and one loggerhead
turtle.

Information gained by tracking the turtles will reveal inter-nesting movements
and the number of times the turtles nest during the season, as well as helping
to identify foraging grounds, migration routes and potential threats.

Pilbara Region conservation officer Joanne King said it was fantastic to involve
the students in the initiative.

\begin{wrapfigure}{l}{0.3\textwidth}%
\includegraphics[width=\linewidth]%
    {{ STATIC_ROOT }}/img/sabrina.jpg}
\caption*{Sabrina Fossette balancing work, life, and baby Nora.}
\end{wrapfigure}

"Students have been tracking the turtles on seaturtle.org and were excited to
discover where the turtles have travelled," she said.

"Casper, a turtle named by Wickham Primary School Maths and Science Centre, has
travelled more than 800km and is currently north of Broome, while Sandy
Shellington, named by Paraburdoo Primary School, travelled from the Montebello
Islands, to the western edge of the Dampier Archipelago, back to the Montes,
across to above Cape Preston, back to the Montes before returning to the
archipelago."
